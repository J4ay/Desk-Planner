\section{Reflektion Lernfortschritt}

\section{Lizenzen}
Wir haben uns ausschließlich für Open Source Lizenzen entschieden. 
Der Grund ist zum einen, dass auch unsere Anwendung später als Open Source vermarktet werden soll und zum anderen die folgenden generellen Vorteile dieser Lizensierung.

\begin{itemize}
    \item	Verbreitung kostenlos
    \item   Entwicklungstempo hoch (bei großer Community)
    \item   Bugs im Code werden schneller gefunden (Mehraugen Prinzip)
    \item	Direkte Modifikation des bestehenden Codes (zur optimalen Eigennutzung)
    \item	Hohe Qualität, da hoher Qualitätsdruck für Entwickler (jeder kann jede Änderung auf den einzelnen Nachverfolgen)
    \item	Meistens gut dokumentiert
    \item	Hohe Innovationspotential (keine Firmenvorgaben etc.)
\end{itemize}

Wie bei allen Dingen im Leben gibt es natürlich auch hier die Kehrseite der Medaille, im Rahmen unseres Projektes haben die Vorteile aber überhand. 
Wir haben folgende Bibliotheken mit der entsprechenden Lizenz verwendet:

\begin{itemize}
\item	MIT Lizenz:
\begin{itemize}
    \item	Material UI
    \item	react-native-date-picker
    \item	Fabric
    \item	React
    \item	Nest
\end{itemize}
\end{itemize}
\begin{itemize}
    \item	Apache Lizenz:
    \begin{itemize}
\item	Keycloak
\item	MongoDB
\end{itemize}
\end{itemize}


\section{Ausblick}
Gegen ist uns das “unsaubere“ Datenbankmodell auf die Füße gefallen. 
In Zukunft werden wir bereits im Vorfeld auf die systematisch einheitliche Namensvergabe achten. 
Dadurch kann allein variablen Namen direkt auf Eigenschaft zurückgeschlossen werden.
 Dies führt zu erheblicher Zeitersparnis bei “Fremd-Codes“. Wir konnten nicht alle Features des Mockups Implementieren. \\
 Darunter zählen:
\begin{itemize}
    \item	Modifikation der Buchungszeit nach dem setzten der Endzeit
    \item	Darstellung der vorhandenen Räume in der Übersicht
    \item	Die Rollenzuweisung der Rechte in einem Bestimmt Raum um gültige buchen zu setzen 
    \item	Live-Chat Funktion  
\end{itemize}
