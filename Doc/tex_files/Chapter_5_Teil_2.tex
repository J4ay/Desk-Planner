\section{Reflektion: Lernfortschritt}
Eine elementare Schnittstelle zwischen den Einzelentwicklern sind Softwareentwicklungsprojekte zur Versionsverwaltung. 
Im Rahmen unsers Projektes haben wir zum (Merg‘en, Pull‘en, Push’en, Commit’en etc.) ausschließlich GitHub verwendet. 
Solche Programme sind unerlässlich, um progressive Fortschritte zu vermerken, dies haben wir besonders im Rahmen dieses Projektes erfahren. 
Wir haben durch eigenständige Einarbeitung und gegenseitigen Austausch, unsere Fähigkeiten in den Scriptsprachen JavaScript bzw./und TypeScript ausgebaut. 
Verstanden was das Front und Backend macht und diese entwickelt und z. B. React und REST API Anbindungen im selbigen Bereich implementiert. 
Wir haben gelernt welche Schritte notwendig sind, um die Architektur zur Projektumsetzung zu designen.
Wir haben gelernt wie eine Benutzeroberfläche mit der Bibliothek Material UI zu entwerfen ist. 
Zur Termingerechten Fertigstellung unseres Vorhabens haben wir mit Milestones und Issues gearbeitet. 
Uns ist nun klar, dass trotz sorgfältiger Planung immer Puffer einzuplanen sind, um realistische Fertigstellungstermine zu prognostizieren. 
Wir sind uns alle aber dennoch einig, dass das wichtigste ein belastbares, kompetentes und kommunikatives Team ist. 
Trotz der hohen Güte des Teams ist ein sauberes Datenbankmodell Elementar, dies haben wir durch ein Negativbeispiel am eigenen Leib erfahren.
Wir haben aus diesem Grund ein Freiheitdatenmodell nutzen müssen. 
In Zukunft werden wir bereits im Vorfeld auf die systematische und einheitliche Namensvergabe der Variablen achten. 
Dadurch kann allein durch den Variablennamen direkt auf Eigenschaft zurückgeschlossen werden.
Dies führt zu erheblicher Zeitersparnis bei “Fremd-Codes“.


\section{Lizenzen}
Wir haben uns ausschließlich für Open Source Lizenzen entschieden. 
Der Grund ist zum einen, dass auch unsere Anwendung später als Open Source vermarktet werden soll und zum anderen die folgenden generellen Vorteile dieser Lizensierung.

\begin{itemize}
    \item	Verbreitung kostenlos
    \item   Entwicklungstempo hoch (bei großer Community)
    \item   Bugs im Code werden schneller gefunden (Mehraugen Prinzip)
    \item	Direkte Modifikation des bestehenden Codes (zur optimalen Eigennutzung)
    \item	Hohe Qualität, da hoher Qualitätsdruck für Entwickler (jeder kann jede Änderung auf den einzelnen Nachverfolgen)
    \item	Meistens gut dokumentiert
    \item	Hohe Innovationspotential (keine Firmenvorgaben etc.)
\end{itemize}

Wie bei allen Dingen im Leben gibt es natürlich auch hier die Kehrseite der Medaille, im Rahmen unseres Projektes haben die Vorteile aber überhand. 
Wir haben folgende Bibliotheken mit der entsprechenden Lizenz verwendet:

\begin{itemize}
\item	MIT Lizenz:
\begin{itemize}
    \item	Material UI
    \item	react-native-date-picker
    \item	Fabric
    \item	React
    \item	Nest
\end{itemize}
\end{itemize}
\begin{itemize}
    \item	Apache Lizenz 2.0:
    \begin{itemize}
\item	Keycloak
\item	MongoDB
\end{itemize}
\end{itemize}


\section{Ausblick}
 Wir konnten nicht alle Features des Mockups Implementieren. \\
 Darunter zählen:
\begin{itemize}
    \item	Modifikation der Buchungszeit nach dem setzen der Endzeit
    \item	Darstellung der vorhandenen Räume in der Übersicht
    \item	Die Rollenzuweisung der Rechte in einem Bestimmt Raum um gültige buchen zu setzen 
    \item	Live-Chat Funktion  
\end{itemize}
