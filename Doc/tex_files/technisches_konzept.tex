\section{Technisches Konzept}

\subsection{Verwendete Frameworks}

\begin{table}[!h]
    \centering
    \begin{tabular}{|l|l|} 
    \hline
    NestJS                                                                       & ExpressJS                                                                    \\ 
    \hline
    NodeJS Framwork                                                              & NodeJS Framework                                                             \\ 
    \hline
    + Durch vorgegebene Struktur lässt sich das Projekt eher strukturiert halten & - Projekt wird bei großer Projektgröße schnell unstrukturiert                \\ 
    \hline
    + Durch die strukturierte Arbeitsweise gut für Teams geeignet                & - Durch die Freiheit bei Technologiewahl eher für einzelne Developer besser  \\ 
    \hline
    - weniger Freiheit bei Implementation                                        & + Mehr Freiheit bei der Implementation durch weniger Strukturvorgaben        \\ 
    \hline
    - Aufgrund strengerer Strukturvorgaben schwerer zu lernen                    & + Durch einfachen Aufbau leicht zu erlernen                                  \\ 
    \hline
    - relativ neue Technologie mit weniger Nutzern                               & + Viele Nutzer und dadurch auch viel Dokumentation                           \\
    \hline
    \end{tabular}
    \end{table}

NestJS war eine Vorgabe des Product Owners (IT Designers Gruppe) für die Entwicklung einer Back-End API. 
Es ist ein Framework das auf der JavaScript Umgebung NodeJS aufbaut und den Code in Module, Controller und Services aufteilt und somit eine klare Struktur zur Entwicklung vorgibt.\\
Aufgrund diesen strengen Strukturvorgaben, sind NestJS Anwendungen, im Vergleich mit bspw. ExpressJS, sehr gut skalierbar.



\begin{table}[!h]
    \centering
    \begin{tabular}{|l|l|} 
    \hline
    React                                                                      & Angular                                                                    \\ 
    \hline
    JS Library                                                             & JS Library                                                           \\ 
    \hline
    + Vorbereitung für Praxissemester (Wird mind. 1 Teammitglied sicher verwenden) & - Nicht sicher im Praxissemester verwendet                \\ 
    \hline
    + Sehr einfaches erstellen von GUIs                & - Erstellen von Benutzeroberflächen etwas komplizierter  \\ 
    \hline
    - weniger strukturiert                                        & + Schon sehr ähnlich zu NestJS strukturiert        \\ 
    \hline
    + mehr Freiheit bei Implementation                    & - weniger Freiheit bei Implementation                                  \\ 
    \hline
    - aufgrund weniger Strukturvorgaben schwerer große Projekte sauber zu halten                               & + aufgrund der erzwungenen Struktur ergeben sich auch bei großen Projekten wenig Probleme                           \\
    \hline
    \end{tabular}
    \end{table}

Einige unserer Teammitglieder haben vor, im Praktikumssemester React zu nutzen. 
Außerdem ist React im Vergleich zu Angular noch etwas einfacher zu lernen.
Daher fiel auch die Wahl auf React, trotz der sehr NestJS-ähnlichen Struktur die Angular bietet.

\subsection{Softwarearchitektur}

\subsection{Datenbanken}