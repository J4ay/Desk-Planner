\chapter{Finale Dokumentation}

In diesem letzten Chapter werden Fazite bezüglich der eigenen Lernfortschritte im Bezug auf das Projekt als auch die Nutzung der Software thematisiert.

\section{Installations- und Administrationshandbuch}

In dieser Sektion werden das Installations- und Administrationshandbuch unterteilt.
\subsection{Installationshandbuch}


\subsection{Administrationshandbuch}
% Behandelt Systemverwaltungsaufgaben wie Wartung, Überwachung und Anpassung eines neu installierten Systems.
Als Administrator des IT-Systems ist seine hauptsächliche Aufgabe die Verwaltung von Gebäuden, Stockwerken und Räumen im Layout Designer.
In diesem kann der Administrator einen zuvor definierten Bereich anhand der Kugel-Komponenten editieren, um das richtige Raumlayout zu erstellen.
Außerdem ist es ihm möglich mit einem Doppel-Klick einen Tisch zu erstellen und diesen beliebig im Raum zu verteilen.
Hier kann der Administrator den Raum erstellen und angeben wo dieser sich befindet, um den Arbeitnehmern im Buchen Bereich vorgefertigte Räume zu geben, welche dann gesucht und gebucht werden können.

\pagebreak

\section{Reflektion: Aufteilung des Teams}

\begin{figure}[!h]
    \centering
    \includegraphics[width=0.8\textwidth]{Aufgabenverteilung.png}
    \caption{Aufgabenverteilung - Erste Präsentation}
    \label{fig:Aufgabenverteilung}
\end{figure}

In der Abbildung zu erkennen, ist die erste Aufgabenverteilung im Vorfeld des Projekts.
Die Aufteilung fand grob in Front-End und Back-End statt, mit kleinen Aufgabenspezifikationen wie Design, GitHub und Dokumentation.
Diese Aufteilung beruht auf vergangene Projekte im Rahmen des Studiums.

Nachdem das Projekt begonnen wurde, wurde die Aufteilung weiter ausgebaut.
Im Vorfeld war klar, dass Rico Hofmann als Front-End Entwickler fungiert, da seine vergangenen Aufgaben rund um das Thema Design vorangegangene Projekte prägten.
Rico Hofmann ist somit der Lead Designer des Teams, denn er baute das Grundgerüst für die Design-Aufgaben, welche später von Darios Pachtsinis und Pinar Gökcek erweitert und vollendet wurden.
Des Weiteren ist Rico der Experte für Material UI und führte außerdem auch das Protokoll für Teambesprechungen mit unserem Betreuer.

Kyle Mezger ist einer der Back-End Entwickler, welcher sich zur Aufgabe machte, die REST API zu erstellen und auszubauen und den Login via Keycloak zu verwalten.
Er hatte sich allerdings auch um die Kommunikation mit dem Betreuer als auch für die wöchentliche Aufgabenverteilung innerhalb der Gruppe via issues in GitHub gekümmert.

Jay Imort hingegen ist der andere Back-End Entwickler, welcher seinen Fokus hauptsächlich auf Nest.js, Routing legt.
Des Weiteren beschäftigte er sich mit dem Layout Designer, in welchem der Benutzer einen Raumlayout erstellen, bearbeiten und in die Datenbank laden kann.

Die Aufgaben im Back-End Bereich wurden meist vermischt und deswegen teilten sich Kyle Mezger und Jay Imort die Aufgaben gerecht und deren Fachwissen entsprechend auf.

Darios Pachtsinis wurde ursprünglich dem Back-End Team zugeteilt, allerdings hatte das Team mehr Designer gebraucht, weswegen sein Aufgabenfeld doch zum Front-End zugeteilt worden ist.
Er hatte Aufgaben erledigt wie das finalisieren von Design Entscheidungen, hauptsächlich rund um den Buchungen Bereich (Booking, etc.).

Pinar Gökcek ist ebenfalls im Front-End zuständig gewesen und hatte Design Entscheidungen von Rico Hofmann finalisiert.
Außerdem erledigte sie Aufgaben speziell im Nachrichten Bereich (Messages, etc.), wie das Empfangen von Nachrichten, als auch dessen Verknüpfung mit der Datenbank.

Zu guter Letzt jedoch war das Team kein Zusammenschluss von 5 Einzelpersonen mit Fachwissen in besonderen Bereich, sondern das Team unterstützte sich gegenseitig, sodass jeder auch außerhalb des zuvor genannten Aufgabenbereichs Aufgaben erledigen konnte.
Dies bedeutet, dass das Team sich immer helfen konnte, sich gegenseitig motiviert hat und auf die Erledigung der Aufgaben anderer achtete und tatkräftig unterstützt hatte.
Die Aufgabenverteilung ist somit grob und soll verdeutlichen, dass die Übergänge zu anderen Aufgabenbereichen und Teammitgliedern vermischen.

\section{Reflektion: Projektmanagement}

Als Projektmanagementsprinzip wurde eine Mischung aus Kanban und Scrum, auch Scrumban, genutzt.
Wie im Voraus erwartet, hatte uns das Kanban Board eine große Hilfe geleistet beim Überblicken der Projektaufgaben.
Das Kanban Board wurde mit einem von GitHub bereitgestellten Issue-System verbunden, welches wir nutzten, um größere Aufgabenteile zu unterteilen und diese einem Mitglied zuzuweisen.

Mithilfe der wöchentlichen Meetings konnten wir uns gut koordiniert auf die Aufgaben des Boards konzentrieren und diese auch größtenteils bewältigen.
Wir setzten unser wöchentliches Meeting auf den Dienstag, um den Montag mit unserem Betreuer zu nutzen.
Die Vorstellung unseres jeweiligen Standes, gab uns mindestens eine weitere Expertenmeinung, welche wir für unser Gruppenmeeting nutzten.
Um diese Expertenmeinung bestmöglich zu nutzen, haben wir ebenfalls für jedes Treffen ein Protokoll geführt, welches wir direkt in unsere Aufgabenbesprechungen untergebracht hatten.


