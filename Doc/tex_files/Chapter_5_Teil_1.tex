\chapter{Finale Dokumentation}

In diesem letzten Chapter werden Fazite bezüglich der eigenen Lernfortschritte im Bezug auf das Projekt als auch die Nutzung der Software thematisiert.

\section{Installations- und Administrationshandbuch}

In Kapitel 5.1 werden das Installations- und Administrationshandbuch unterteilt.
\subsection{Installationshandbuch}

\subsection{Administrationshandbuch}
% Behandelt Systemverwaltungsaufgaben wie Wartung, Überwachung und Anpassung eines neu installierten Systems.

\section{Aufteilung des Teams}

Rico: Designing 
Kyle: Erstellen / Ausbauen der REST API
Jay: MongoDB aufsetzen
Darios: Buchungen funktional und Design gemacht
Pinar: Messages funktional und Design gemacht

\section{Reflektion Projektmanagement}

Als Projektmanagementsprinzip wurde eine Mischung aus Kanban und Scrum, auch Scrumban, genutzt.
Wie im Voraus erwartet, hatte uns das Kanban Board eine große Hilfe geleistet beim Überblicken der Projektaufgaben.
Das Kanban Board wurde mit einem von GitHub bereitgestellten "Issue"-System verbunden, welches wir nutzten, um größere Aufgabenteile zu unterteilen und diese einem Mitglied zuzuweisen.

Mithilfe der wöchentlichen Meetings konnten wir uns gut koordiniert auf die Aufgaben des Boards konzentrieren und diese auch größtenteils bewältigen.
Wir setzten unser wöchentliches Meeting auf den Dienstag, um den Montag mit unserem Betreuer zu nutzen.
Die Vorstellung unseres jeweiligen Standes, gab uns mindestens eine weitere Expertenmeinung, welche wir für unser Gruppenmeeting nutzten.
Um diese Expertenmeinung bestmöglich zu nutzen, haben wir ebenfalls für jedes Treffen ein Protokoll geführt, welches wir direkt in unsere Aufgabenbesprechungen untergebracht hatten.


