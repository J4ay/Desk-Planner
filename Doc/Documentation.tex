% Preamble
% Document type
\documentclass{article}

% Additional packages
\usepackage[utf8]{inputenc}

% Clickable Links in TOC
\usepackage[hidelinks]{hyperref}

% Images/ figures package
\usepackage{graphicx}
\graphicspath{ {./images/} }

% Language
\usepackage[ngerman]{babel}

% Information
\title{Projekt DeskPlanner}
\author{Pinar Gökcek, Kyle Mezger, Jay Imort, Rico Hofmann, Darios Pachtsinis}
\date{04. April 2022}

% Content
% Beginning of document
\begin{document}

% Titlepage
\begin{titlepage}
    \centering
    % Output title
    \maketitle

    % Fill rest of page
    \vfill

\end{titlepage}

% Table of contents
\tableofcontents

\pagebreak

\section{Zielgruppe, Problem, Eigenschaften, Alleinstellungsmerkmal}

\subsection{Zielgruppe}

\subsection{Problem}

\subsection{Eigenschaften}
Die Web-Applikation DeskPlanner wird mindestens folgende Eigenschaften nachweisen können:
<<<<<<< HEAD
=======

% missing table, which concluded in errors

>>>>>>> 266c402d18a5004983b86d767e861b36064694d1
Des Weiteren sind kleine Eigenschaften wie Einhaltung der 7 Usability Prinzipien
vorgesehen, um so die Bedienung des Kunden so einfach wie möglich zu halten. 
Beispiele hierfür sind Hilfstexte, welche die verschiedenen Buttons erklären sollen
oder selbsterklärende Icons der jeweiligen Buttons. Außerdem soll der Nutzer nicht
mit Information beschüttet werden, sondern eine auf seinen Nutzungskontext
angepasste Benutzeroberfläche sorgen für Aufgabenangemessenheit. Durch die 
individuelle Raumgestaltung erhält der Nutzer ein steuerbares Softwaresystem,
welches genau das machen wird, was der Nutzer auch erwartet.

\subsection{Alleinstellungsmerkmal}
Als herausragendes Leistungsmerkmal des DeskPlanners bezeichnet man die 
effiziente Arbeitsplatznutzung. In der heutigen Branche, vor allem durch Corona 
bestätigt, gilt es, die Kosten immer weiter zu reduzieren. Durch die 
Raumbuchungsmöglichkeiten, kann ein Unternehmen die Arbeitsplätze minimieren, Home Office
Mitarbeiter perfekt einplanen und dadurch Kosten der Technik und Größe des Raumes
einsparen für andere Investitionen. Als Open Source Web-Applikation kann man
als Unternehmen sogar lizenzfrei den Arbeitsplatz kosteneffizienter nutzen.
Genau hier soll der DeskPlanner vor allem kleineren Unternehmen unterstützen, um 
auch in Zeiten der Pandemie nicht insolvent zu gehen.

\section{User Stories}

\section{User Interface Entwürfe}

\section{Technisches Konzept}

\subsection{Verwendete Frameworks}

\subsection{Softwarearchitektur}

\subsection{Datenbanken}

% Bibliography
% \bibliography{file.bibtex}
\bibliographystyle{plain}

% Ending of document
\end{document}
