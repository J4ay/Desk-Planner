% Preamble
% Document type
\documentclass{article}

% Additional packages
\usepackage[utf8]{inputenc}
\usepackage{wrapfig}
% Clickable Links in TOC
\usepackage[hidelinks]{hyperref}

% Images/ figures package
\usepackage{graphicx}
\graphicspath{ {./images/} }

% Language
\usepackage[ngerman]{babel}

% Information
\title{Projekt DeskPlanner}
\author{
Gökcek, Pinar\\
\texttt{pigoit00@hs-esslingen.de}
\and
Mezger, Kyle\\
\texttt{kymeit00@hs-esslingen.de}
\and
Imort, Jay\\
\texttt{jaimit00@hs-esslingen.de}
\and
Hofmann, Rico\\
\texttt{rihoit00@hs-esslingen.de}
\and
Pachtsinis, Darios\\
\texttt{dapait03@hs-esslingen.de}
}
\date{04. April 2022}

% Content
% Beginning of document
\begin{document}

% Titlepage
\begin{titlepage}
    \centering
    % Output title
    \maketitle

    % Hide page number
    \thispagestyle{empty}

    % Fill rest of page
    \vfill

\end{titlepage}

% Table of contents
\tableofcontents

\pagebreak

% \include{./tex_files/}
\chapter*{Zeitplan}


\begin{sidewaysfigure}[!ht]
    \centering
    \includegraphics[width=0.6\textwidth]{Zeitplan_DeskPlanner.png}
    \caption{Zeitplan}
    \label{fig:Zeitplan}
\end{sidewaysfigure}

Der Zeitplan hangelt sich an den vorgegebenen Meilensteinen 1-4 entlang.\\

\textbf{Meilenstein 1}\\
Präsentation:
\begin{itemize}
    \item Aufgabenverteilung
    \item Projektmanagement
    \item Technologien
    \item Funktionsumfang
    \item UI-Entwürfe, z.B. Low Fidelity Mock-Ups
\end{itemize}

\textbf{Meilenstein 2}\\
Dokumentation:
\begin{enumerate}
    \item Zielgruppe, Problem, Eigenschaften, Alleinstellungsmerkmal
    \item User Stories
    \item User Interface Entwürfe
    \item Technisches Konzept
    \begin{itemize}
        \item Verwendete Frameworks
        \item Softwarearchitektur
        \begin{itemize}
            \item UML-Verteilungsdiagramm des Gesamtsystems
            \item UML-Komponentendiagramm (jeweils für Client und Server)
        \end{itemize}
        \item Datenbank
        \begin{itemize}
            \item Entity Relationship Model
        \end{itemize}
    \end{itemize}
    \item UI-Entwürfe, z.B. Low Fidelity Mock-Ups
\end{enumerate}

\textbf{Meilenstein 3}

\textbf{Meilenstein 4}

\section{Zielgruppe, Problem, Eigenschaften, Alleinstellungsmerkmal}

\subsection{Zielgruppe}

\subsection{Problem}
Während der Pandemiezeit wird verstärkt von Zuhause aus gearbeitet, dadurch schwankt die Anzahl der Arbeiter im Büro stark und die interdisziplinären Kollegen habe keinen Überblick mehr darüber, wer im Büro ist und wer nicht.
Zumal grundsätzlich nicht jeder in das Home-Office kann bzw. will, dies kann diverse Gründe, wie z. B. fehlende Räumlichkeiten oder die geschwächte Konzentrationsfähigkeit in den eigenen vier Wänden sein.
Agile Arbeitsplätze haben den Vorteil, dass der Arbeitsplatz je nach momentanem empfinden gewählt werden kann z. B. im sogenannten Ruhebereich, Bereiche die näher an Heizkörpern sind oder hellere Arbeitsplätze an Fenstern.
Projektspezifisch kann es vorteilhaft sein neben einem bestimmten Kollegen zu arbeiten, um den Austausch produktiver zu gestalten.
Diese Bereiche können frei benutzt werden, sofern diese nicht durch eine andere Kollegin oder einen anderen Kollegen belegt sind.
Derzeit ist es gängig, dass die Bereiche nach dem Prinzip “first come first serve“ besetzt werden. 

\paragraph{} Wenn man am Morgen an einem Arbeitsplatz sitzt, dann aber im Laufe des Tages für einen längeren Zeitraum diesen verlassen muss, wird diese Ressource unnötig blockiert.
Raumbelüftungssysteme, Heizungen und Klimaanlagen für Räume bzw. Gebäude werden nicht nach der Personenauslastung betrieben, in Hinblick auf die steigenden Energiekosten, ist dies weder ökonomisch noch nachhaltig.
Immungeschwächte Personen die ihre Tätigkeit nur vor Ort im Büro erledigen können, haben pandemiebedingt nicht die Möglichkeit Arbeitsplätze zu Buchen, die weiter weg von den restlichen Arbeitsplätzen bzw. Kollegen sind. 
Der Arbeitgeber, der nur eine bestimmte X Arbeitsplätzen hat, kann nicht mehr als X Arbeiter in dem Bürokomplex beschäftigen. 
Desweitern gibt es derzeit keine einfache Möglichkeit die Büro Belegschaft, über einen längeren Zeitraum zu analysieren. 
Diese Analyse kann unter anderem zur dauerhaften Arbeitsplatz Reduktion, bei gleicher Mitarbeiterzahl führen.

\paragraph{} Unsere Anwendung ermöglicht es, die oben angesprochenen Fälle zu Gunsten des Arbeitgebers und -nehmers zu lösen.
Die Raumplanung ermöglichet eine Buchung in 15 min Rastern auch Dauerbuchungen über mehrere Monate sind möglich.
Nicht nur Arbeitsplätze in einem Raum, sondern bestimmte Räume in bestimmten Gebäuden können gebucht werden.
Auch ist eine Chatfunktion zwischen Teamleitern und Mitarbeitern bzw. zwischen Mitarbeitern möglich.
Dies hat den Vorteil das Buchungen bei Sonderfällen unter Mitarbeitern einfach getauscht werden können.
Eine Synchronisation mit dem Outlookkalender wäre auch denkbar.

\subsection{Eigenschaften}
Die Web-Applikation DeskPlanner wird mindestens folgende Eigenschaften \\nachweisen können:

\begin{itemize}
    \item Open Source (Lizenzfrei) als kostenfreie Möglichkeit für vor allem kleine Unternehmen
    \item Simple Möglichkeiten, um Komplexität zu vermeiden, aber so viele Mög-lichkeiten der Raumplanung beizubehalten
    \item Three-Tier Architecture ähnliche Anwendungsweise der Raumplanung im Sinne von Gebäude > Stockwerk > Raum 
\end{itemize}

\paragraph{} Des Weiteren sind kleine Eigenschaften wie Einhaltung der 7 Usability Prinzipien vorgesehen, um so die Bedienung des Kunden so einfach wie möglich zu halten. 
Beispiele hierfür sind Hilfstexte, welche die verschiedenen Buttons erklären sollen oder selbsterklärende Icons der jeweiligen Buttons. 
Außerdem soll der Nutzer nicht mit Information beschüttet werden, sondern eine auf seinen Nutzungskontext angepasste Benutzeroberfläche sorgen für Aufgabenangemessenheit. 
Durch die individuelle Raumgestaltung erhält der Nutzer ein steuerbares Softwaresystem, welches genau das machen wird, was der Nutzer auch erwartet.

\subsection{Alleinstellungsmerkmal}
Als herausragendes Leistungsmerkmal des DeskPlanners bezeichnet man die effiziente Arbeitsplatznutzung. 
In der heutigen Branche, vor allem durch Corona bestätigt, gilt es, die Kosten immer weiter zu reduzieren. 
Durch die Raumbuchungsmöglichkeiten, kann ein Unternehmen die Arbeitsplätze minimieren, Home Office Mitarbeiter perfekt einplanen und dadurch Kosten der Technik und Größe des Raumes einsparen für andere Investitionen. 
Als Open Source Web-Applikation kann ein Unternehmen sogar lizenzfrei den Arbeitsplatz kosteneffizienter nutzen.
Genau hier soll der DeskPlanner vor allem kleineren Unternehmen unterstützen, um auch in Zeiten der Pandemie nicht insolvent zu gehen.


<<<<<<< HEAD
\subsection{Problemstellung}
Während der Pandemiezeit wird verstärkt von Zuhause aus gearbeitet, dadurch schwankt die Anzahl der Arbeiter im Büro
stark und die interdisziplinären Kollegen habe keinen Überblick mehr darüber, wer im Büro ist und wer nicht.
Auch die Geschäftsleitung hat keine direkte Möglichkeit, ohne die entsprechenden Teamleiter zu kontaktieren, sich über 
die Momentane Anwesenheit zu informieren.
Zumal grundsätzlich nicht jeder in das Home-Office kann bzw. will, dies kann diverse Gründe, wie z. B. fehlende
Räumlichkeiten oder die geschwächte Konzentrationsfähigkeit in den eigenen vier Wänden sein. Agile Arbeitsplätze
haben den Vorteil, dass der Arbeitsplatz je nach momentanem empfinden gewählt werden kann z. B. im sogenannten Ruhebereich,
Bereiche die näher an Heizkörpern sind oder hellere Arbeitsplätze an Fenstern. Projektspezifisch kann es vorteilhaft sein
neben einem bestimmten Kollegen zu arbeiten, um den Austausch produktiver zu gestalten. Diese Bereiche können frei benutzt werden,
sofern diese nicht durch eine andere Kollegin oder einen anderen Kollegen belegt sind. Derzeit ist es gängig, dass die Bereiche nach
dem Prinzip “first come first serve“ besetzt werden.\\\\
Wenn eine Person am Morgen an einem Arbeitsplatz sitzt, dann aber im Laufe des Tages für einen längeren Zeitraum diesen verlassen muss,
wird diese Ressource unnötig blockiert. Raumbelüftungssysteme, Heizungen und Klimaanlagen für Räume bzw. Gebäude werden nicht
nach der Personenauslastung betrieben, in Hinblick auf die steigenden Energiekosten, ist dies weder ökonomisch noch nachhaltig.
Theortisch ist es auch möglich durch die hierdurch gewonnene Statistik, die Raum- bzw. Gebäudeausnutzungen zu optimieren
und dadurch Mietkosten einzusparen.
Immungeschwächte Personen die ihre Tätigkeit nur vor Ort im Büro erledigen können, haben pandemiebedingt nicht die Möglichkeit
Arbeitsplätze zu Buchen, die weiter weg von den restlichen Arbeitsplätzen bzw. Kollegen sind. Der Arbeitgeber, der nur eine bestimmte
X Arbeitsplätzen hat, kann nicht mehr als X Arbeiter in dem Bürokomplex beschäftigen. Desweitern gibt es derzeit keine einfache Möglichkeit
die Büro Belegschaft, über einen längeren Zeitraum zu analysieren. Diese Analyse kann unter anderem zur dauerhaften Arbeitsplatz Reduktion,
bei gleicher Mitarbeiterzahl führen.\\\\
Unsere Anwendung ermöglicht es, die oben angesprochenen Fälle zu Gunsten des Arbeitgebers und -nehmers zu lösen. 
Die Raumplanung ermöglichet eine Buchung in 15 min Rastern auch Dauerbuchungen über mehrere Monate sind möglich. 
Nicht nur Arbeitsplätze in einem Raum, sondern bestimmte Räume in bestimmten Gebäuden können gebucht werden. 
Auch ist eine Chatfunktion zwischen Teamleitern und Mitarbeitern bzw. zwischen Mitarbeitern möglich. 
Dies hat den Vorteil das Buchungen bei Sonderfällen unter Mitarbeitern einfach getauscht werden können. 
Eine Synchronisation mit dem Outlookkalender wäre auch denkbar.
=======
\section{User Stories}

Mitarbeiter/-in:

\begin{itemize}
    \item Als Mitarbeiter/-in möchte ich Sitzplätze anschauen, damit ich in einem Blick sehe welche Plätze verfügbar/vergeben sind.

    \item Als Mitarbeiter/-in möchte ich Sitzplätze buchen, damit ich für die gebuchte Zeit einen Arbeitsplatz zur verfügung habe.

    \item Als Mitarbeiter/-in möchte ich Sitzplätze abonnieren, damit ich einen Sitzplatz über längere Zeit zu einer gewissen Zeit reserviert habe ohne jeden Tag einzeln buchen zu müssen.

    \item Als Mitarbeiter/-in möchte ich Sitzplätze stornieren, damit ich bei einer Fehlbuchung oder Krankheit den Platz für andere Mitarbeiter wieder buchbar machen kann.

    \item Als Mitarbeiter/-in möchte ich mit einem Platz Besitzer chatten können, damit ich anfragen kann ob der jeweilige Platz verfügbar gemacht werden könnte.

    \item Als Mitarbeiter/-in möchte ich Nachrichten erhalten, damit ich bei der Anfrage eines anderen Mitarbeiters meinen gebuchten Platz für diesen stornieren kann.
\end{itemize}

\noindent Admin:

\begin{itemize}
    \item Als Administrator/-in möchte ich alle Funktionen des Mitarbeiters haben, da ich mitunter Mitarbeiter bin.

    \item Als Administrator/-in möchte ich das Layout von Räumen anpassen, damit es dem tatsächlichen Büro entspricht und es somit besser erkennbar ist für Mitarbeiter um welchen Raum und Sitzplatz es sich handelt.

    \item Als Administrator/-in möchte ich Gruppen verwalten können, damit ich Bestimmte Positionen und Personen in Gruppen einteilen kann, damit ein Geschäftsführer diese in Raumbuchungen beschränken kann.
\end{itemize}

\noindent Geschäftsführer/-in:

\begin{itemize}
    \item Als Geschäftsführer/-in möchte ich alle Funktionen des Mitarbeiters haben, da ich mitunter Mitarbeiter bin.

    \item Als Geschäftsführer/-in möchte ich die Buchbarkeit von Räumen auf Gruppen einschränken, damit bestimmte Räume für bestimmte Positionen/Personen ausgelegt sind.
\end{itemize}
\begin{figure}
    \centering
    \includegraphics[width=0.8\textwidth]{UseCase_Diagram.png}
    \caption{Usecase-Diagramm}
    \label{fig:Usecase-Diagramm}
\end{figure}

\chapter{User Interface Entwürfe}

Die User Interfaces wurden so gestaltet, dass sie die zuvor genannten User Stories, Use Cases und Anforderungen bestmöglich erfüllen. 
\paragraph{}Dabei werden die Mitarbeiter Use Cases in einer mobile first Anwendung verpackt, weil die Arbeitsplatzbuchungen schnell und leicht von der Hand gehen sollen.
Ebenfalls ist mobile first praktisch für die Nachtrichten-Funktion, da die Erreichbarkeit und die Möglichkeit schnell zu antworten, auf mobilen Geräten besser ist als auf Desktop Systemen. 
\paragraph{}Die Use Cases des Administrators und der Geschäftsführung werden wiederum auf Desktop-Systeme ausgelegt, da dort eine übersichtliche, effiziente und präzise Nutzung wichtiger ist, als die Flexibilität auf das System zuzugreifen zu können. 

\newpage

\section{Mitarbeiter User Interface}

\begin{wrapfigure}[18]{r}{5.5cm}
  \includegraphics[width=5.5cm]{sketchBooking.png}
  \caption{User Interface: Buchen}
\end{wrapfigure}

Nachfolgend werden die Funktionen und Seiten des mobile first User Interfaces visualisiert und erläutert. 

\subsection{Buchen}

Sobald die Anwendung abgerufen wird, erscheint das Buchen-Fenster.
\paragraph{}Ersichtlich ist das aktuell geöffnete Fenster durch die Funktionsleiste am unteren Ende der Anwendung. 
Dort wird die ausgewählte Funktion, sprich das aktuelle Fenster, durch die dunkle schwarze Färbung und den Balken darunter hervorgehoben.
\paragraph{}Im oberen Teil der Anwendung ist es möglich über Dropdown-Menüs das Gebäude, die Etage und den Raum rauszusuchen, in dem die Buchung gewünscht ist.
Dabei geschieht die Auswahl von rechts startend aus, spricht zuerst das Gebäude, dann die Etage und zuletzt den Raum.
Dies ist wichtig, weil je nach Gebäude andere Etagen bestehen, welche dann wieder nur bestimmte Räume enthalten.

\paragraph{}Wenn im oberen Teil alle Angaben bis zum Raum hin gegeben wurden, erscheint in der Mitte der Anwendung der ausgewählte Raum.
Dieser wird durch ein Rechteck mit dünnen schwarzen Linien in der Anwendung begrenzt.
In diesem Rahmen soll es möglich sein den Raum zu verschieben und zu vergrößern und zu verkleinern.
Der Raum an sich besteht zum einen aus Wänden, dicke schwarze Striche, und Türen, grau gestrichelte Einsätze in den Wänden. Türen und Wände werden 
lediglich zur Orientierung und besseren Vorstellung des Raumes genutzt. 
\paragraph{}In dem Raum befinden sich schließlich farbige Rechtecke, welche Arbeitsplätze darstellen, welche zum Interagieren benutzt werden. Die farbliche Kennzeichnung beschreibt, wie der Platz an dem heutigen Tag genutzt wird.
Blau zeigt an, dass der Platz frei verfügbar ist für heute, rot zeigt an, dass der Platz mindestens in einem Zeitfenster von heute eine Buchung eines anderen Arbeitskollegen vorliegen hat und grün zeigt an, dass der Arbeitsplatz für heute selbst in mindestens einem Zeitfenster gebucht ist.
\paragraph{}Um weitere und genauere Informationen über die Belegung eines Arbeitsplatzes zu erhalten, wird dieser angeklickt .

\begin{wrapfigure}[21]{l}{5.5cm}
  \includegraphics[width=5.5cm]{sketchTime1.png}
  \caption{User Interface: Buchen - Zeitauswahl}
\end{wrapfigure}

\paragraph{}Mit dem Anklicken erscheint ein Pop-Up Fenster, für die Zeiteingabe. 
In der Mitte des Fensters befindet sich die Zeitangabe, die aktuell ausgewählt ist. 
Durch unabhängiges hoch- und runterwischen des Datums, der Stunden und Minuten verändert sich die Zeitangabe.
Die Zeitangaben sind dabei in 15-Minutenschlitze unterteilt.

\paragraph{} Zu Beginn startet das Pop-Up mit der "`Von"' Zeitangabe, welche den Start des Buchungszeitraumes festlegt. 
Im oberen Teil des Pop-Ups wird durch blaue Farbe hervorgehoben, ob es sich um die Eingabe der Startzeit oder der Endzeit handelt.
Mit einem Klick auf "`Von..."' oder "`Bis..."' lassen sich somit auch die Zeiteingaben dafür wechseln. 

\paragraph{}Im unteren Teil des Pop-Ups sind stets die Eingaben verwerfbar durch einen Klick auf "`Abbrechen"'.
Rechts daneben sind die Eingaben zu bestätigen und abzuschicken durch einen Klick auf "`Buchen"'.
Dabei ist der Buchen-Button bei nur einer "`Von"' Angabe noch ausgegraut und nicht anklickbar.
Erst nach der Auswahl der Startzeit und der Endzeit kann der Buchen-Button geklickt werden.

\begin{figure}[!h]
  \centering
  \includegraphics[width=1\textwidth]{sketchTime23.png}
  \caption{User Interface: Buchen - Informationen aus der Zeitauswahl}
  \label{fig:sketch_time_23}
\end{figure}

\paragraph{}Aus der Zeitauswahl folgen noch weitere Informationen dazu ob ein Arbeitsplatz zu einem Zeitpunkt von jemand Anderen gebucht, frei oder selbst gebucht ist. \\
Schwarze Zeitangaben sind frei verfügbar und können ohne Weiteres gebucht werden. \\
Grüne Zeitangaben symbolisieren eigene Buchungen im System. \\
Diese dienen lediglich zur eigenen Information und besitzen keine weiteren Interaktionsmöglichkeiten.\\
Rote Zeitangaben sind von einem Kollegen gebucht, dieser Kollege wird auch in dem Pop-Up Fenster angezeigt.
Anstatt dem Buchen-Button kann der Kollege nun per Nachricht angeschrieben werden, falls der Arbeitsplatz beispielsweise persönlich benötigt wird. 

\subsection{Buchungen/Chronik}

\begin{wrapfigure}[20]{r}{5.5cm}
  \includegraphics[width=5.5cm]{sketchBuchungen.png}
  \caption{User Interface: Übersicht getätigte Buchungen}
\end{wrapfigure}

Durch einen Klick auf das Buch in der Funktionsleiste unten, geschieht ein Wechsel in die Übersicht aller getätigten Buchungen.
Die Buchungsübersicht wird in Abbildung 9 dargestellt und wird ebenfalls wieder durch die schwarze Färbung des Symbols in der Funktionsleiste und dem Balken darunter hervorgehoben. \\
\paragraph{}In diesem Fenster werden alle Buchungen als Kacheln angezeigt, welche im System getätigt wurden.
Dabei umfasst jede Kachel den Raum, in dem die Buchung gemacht wurde, das Datum und den Zeitraum, der gebucht wurde. \\
\paragraph{}In diesem Fenster lässt sich auch jede getätigte Buchung wieder stornieren durch einen Klick auf das Minus-Symbol.

\newpage
\subsection{Nachrichten}

\begin{wrapfigure}[23]{l}{5.5cm}
  \includegraphics[width=5.5cm]{sketchNachrichten.png}
  \caption{User Interface: Nachrichtenübersicht}
\end{wrapfigure}

Mit einem Klick auf das Briefsymbol, links in der Funktionsleiste, wechselt die Anwendung zur Nachrichtenübersicht.
Dieses Symbol zeigt auch mit einer kleinen hochgestellten Zahl neben dem Icon an, dass ungelesene Nachrichten eingegangen sind.
\paragraph{}Die Nachrichtenübersicht enthält sämtliche Chats, die selbst gestartet wurden oder an einen persönlich gerichtet sind. 
Denn wie zuvor schon erwähnt, besteht die Möglichkeit, Kollegen eine Nachricht zu schreiben, wenn diese einen Arbeitsplatz gebucht haben, der jedoch selbst benötigt wird. 
Dies geschieht über die Zeitraumeingabe, beim Klicken des "`Benachrichtigen"' Buttons, wenn ein Arbeitsplatz, zur gewünschten Zeit, belegt ist.
\paragraph{} Die Nachrichtenübersicht enthält, wie bei der Buchungsübersicht, ebenfalls alle offenen Chats als Kacheln aufgelistet. 
Diesmal enthalten die Kacheln den Namen des anderen Chatteilnehmers, den Raum in dem die betroffene Buchung ist und den Textanfang der neusten Nachricht im Chat.
Chatkacheln mit ungelesenen Nachrichten werden zudem noch hervorgehoben.

\paragraph{} Für mehr Übersichtlichkeit in der Nachrichtenübersicht ist der Chat löschbar.
Dieser sollte aber lediglich gelöscht werden, wenn das Thema des jeweiligen Chats sich erledigt hat.
Dazu lässt sich jeder Chat löschen, mit einem Klick auf das X-Symbol der jeweiligen Kachel.
Dies löscht den Chat nur persönlich und nicht für den anderen Chatteilnehmer, dieser kann den Chat so lange einsehen, bis er ihn auch löscht.

\newpage
\begin{wrapfigure}[15]{r}{5.5cm}
  \includegraphics[width=5.5cm]{sketchChat.png}
  \caption{User Interface: Beispielchat}
\end{wrapfigure}

Durch das Anklicken einer Kachel in der Nachrichtenübersicht in Abbildung 10, erscheint der Chat inklusive Chatverlauf.
Der Chat ist in Abbildung 11 dargestellt.

\paragraph{} Im Kopf des Chats sind nochmal die gleichen Informationen, wie in der Übersicht, bis auf den Anfang der neusten Nachricht. 
Direkt darunter ist der Chatverlauf.
Eigene Nachrichten sind hierbei nach rechts verschoben und Nachrichten des Gegenüber nach links.

\paragraph{} Oberhalb der Funktionsleiste ist noch die Eingabeleiste, welche durch Anklicken das Verfassen einer Nachricht ermöglicht. 
Die Eingabeleiste öffnet hierbei schließlich die Texteingabe des Gerätes.
Nach fertigem Verfassen der Nachricht kann diese abschickt werden mit dem Symbol rechts neben der Eingabeleiste. 

\vspace{5cm}

\section{Administrator User Interface}

Nachfolgend wird auf das Interface für Administratoren und Geschäftsführer eingegangen. 
Dies wird genutzt um neue Räume im System zu registrieren, zu löschen und zu bearbeiten.
Ebenfalls lassen sich dort Nutzergruppen verwalten, die bestimmen, welcher Mitarbeiten in welchem Raum buchen kann. 

\newpage
\subsection{Raum-Editor}
Die genannten Aufgaben des Administrator User Interfaces lassen sich in dem Editor, in Abbildung 12, vornehmen.

\begin{figure}[!h]
  \centering
  \includegraphics[width=1\textwidth]{sketchEditor.png}
  \caption{User Interface: Raum-Editor}
  \label{fig:sketch_RaumEditor}
\end{figure}

Von oben nach unten durchgehend, werden nachfolgend die Funktionen des Editors erklärt.
\\
Oben links besitzt der Editor Auswahlmöglichkeiten für das Gebäude, die Etage und den Raum.
Ausgewählt wird über Dropdown Menüs, welche von links nach rechts ausgewählt werden müssen, wie in der normalen Raumwahl im Buchen Fenster.
Mit einem Rechtsklick auf Einträge im Dropdown Menü lassen sich Räume löschen.
Über das Plus-Symbol rechts neben dem Dropdown Menü kann ein neuer Eintrag jeweils zum System hinzugefügt werden.
Dabei ist es so, dass immer ein Gebäude hinzugefügt werden kann, da dies nicht von vorherigen Auswahlen abhängt. 
Um eine Etage hinzuzufügen, muss ein Gebäude ausgewählt sein, damit die Etage dem ausgewählten Gebäude zugeordnet werden kann.
Zuletzt kann ein Raum hinzugefügt werden, indem ein Gebäude und eine Etage ausgewählt werden.
\\
Bei sämtlichem Hinzufügen erscheint eine Eingabeleiste für den Namen des Eintrags.
Die Eingabe bestätigt sich schließlich mit Klicken von "`Ok"' und verwerft sich mit dem Klicken des X-Symbols.

\paragraph{} Ist nun einen Raum ausgewählt oder in der Neuerstellung soweit, dass das Layout angepasst werden kann, kann oben rechts die Gruppe auswählen werden, ab welcher der Raum buchbar ist.
Dies geschieht ebenfalls über ein Dropdown Menü und mit dem Plus-Symbol kann eine neue Benutzergruppe hinzugefügt werden. 

\paragraph{}Mit dem Auswählen oder Erstellen eines Raumes, wird dieser auch unterhalb der oberen Leiste angezeigt. 
In dem Rechteck mit den dünnen schwarzen Linien, lässt sich der Raum interaktiv bearbeiten.
\\
Oben rechts in der Bearbeitungsfläche sind die drei Objekte, die sich beliebig oft in den Raum ziehen lassen.
Nämlich Türen, Wände und Arbeitsplätze.
Die Visualisierung von Türen, Wänden und Arbeitsplätzen ist in "`3.1.1 Buchen"' beschrieben
\\
Türen lassen sich auf bestehende Wände draufziehen, skalieren und verschieben.
Wände lassen sich einfach in die Bearbeitungsfläche ziehen und wie Polygonlinienzüge anpassen.
Arbeitsplätze werden in den Raum reingezogen und können dort auch noch verschoben und skaliert werden.
Die Elemente in der Bearbeitungsfläche lassen sich generell anklicken und bearbeiten.
Mit der Entf. Taste lässt sich das angeklickte Element auch löschen.

\paragraph{} Unten rechts sind letztlich die Änderungen an einem Raum komplett verwerfbar durch "`Abbrechen"'
oder eben speicherbar mit "`Speichern"'.
Nach dem Speichern wird der Raum auch direkt ins System eingefügt.

\paragraph{} Unten links ist nun noch ein Button für die Benutzerverwaltung.
Mit einem Klick auf diesen öffnet sich ein Pop-Up, in welchem die Mitarbeiter den Berechtigungsgruppen zugewiesen werden können.

\newpage
\subsection{Benutzerverwaltung}

\begin{wrapfigure}[20]{r}{5.5cm}
  \includegraphics[width=5.5cm]{sketchBenutzerverwaltung.png}
  \caption{User Interface: Benutzerverwaltung Pop-Up}
\end{wrapfigure}

Das Pop-Up für die Gruppenverwaltung und Nutzerzuweisungen in die Gruppen wird in Abbildung 13 beschrieben.
\\
\paragraph{}Hier sind alle Gruppen ersichtlich, die bisher erstellt wurden.
Die Gruppen sind per per Drag and Drop neu sortierbar, damit eine hierarchische Struktur entsteht.
Dabei ist die niedrigste Hierarchiestufe oben und die höchste unten. 
Eine höhere Stufe besitzt dabei jede Zugangsberechtigung der Stufen unterhalb dieser. 
\\
\paragraph{}Die Gruppen können auch aufgeklappt werden, wodurch die jeweiligen Benutzer in dieser Gruppe angezeigt werden.
Diese sind ebenfalls per Drag and Drop in andere Gruppen verschiebbar.
\\
\paragraph{}Nach den getätigten Änderungen sind diese speicherbar oder verwerfbar. 


>>>>>>> e2fd2fd2e6af937f85cef4ac210cc9da5b931ec0

\chapter{Technisches Konzept}

\section{Verwendete Frameworks}

\begin{table}[!ht]
    \centering
    \resizebox{\textwidth}{!}{\begin{tabular}{|l|l|} 
    \hline
    \textbf{NestJS}                                                                       & \textbf{ExpressJS}                                                                    \\ 
    \hline
    \emph{NodeJS Framwork}                                                              & \emph{NodeJS Framework}                                                             \\ 
    \hline
    + Durch vorgegebene Struktur lässt sich  & - Projekt wird bei großer Projektgröße schnell                \\
    das Projekt eher strukturiert halten & unstrukturiert \\
    \hline
    + Durch die strukturierte Arbeitsweise                & - Durch die Freiheit bei Technologiewahl  \\
    gut für Teams geeignet &  eher für einzelne Developer besser \\
    \hline
    - weniger Freiheit bei Implementation                                        & + Mehr Freiheit bei der Implementation        \\
      & durch weniger Strukturvorgaben \\
    \hline
    - Aufgrund strengerer Strukturvorgaben                    & + Durch einfachen Aufbau leicht zu erlernen                                  \\
    schwerer zu lernen & \\ 
    \hline
    - relativ neue Technologie mit weniger Nutzern                               & + Viele Nutzer und dadurch auch viel                           \\
     und Tutorials & Dokumentation und Tutorials \\
    \hline
    \end{tabular}}
    \end{table}

NestJS war eine Vorgabe des Product Owners (IT Designers Gruppe) für die Entwicklung einer Back-End API. 
Es ist ein Framework das auf der JavaScript Umgebung NodeJS aufbaut und den Code in Module, Controller und Services aufteilt und somit eine klare Struktur zur Entwicklung vorgibt.\\
Aufgrund diesen strengen Strukturvorgaben, sind NestJS Anwendungen, im Vergleich mit bspw. ExpressJS, sehr gut skalierbar.

\begin{table}[!h]
    \centering
    \resizebox{\textwidth}{!}{\begin{tabular}{|l|l|}
    \hline
    \textbf{React}                                                                      & \textbf{Angular}                                                                    \\ 
    \hline
    \emph{JS Library}                                                             & \emph{JS Library}                                                           \\ 
    \hline
    + Vorbereitung für Praxissemester  & - Etwas weniger verbreitet als                \\
    da React sehr weit verbreitet ist  & React\\
    \hline
    + Sehr einfaches erstellen von GUIs                & - Erstellen von Benutzeroberflächen etwas  \\
     &  komplizierter\\
    \hline
    - weniger strukturiert                                        & + Schon sehr ähnlich zu NestJS strukturiert        \\ 
    \hline
    + mehr Freiheit bei Implementation                    & - weniger Freiheit bei Implementation                                  \\ 
    \hline
    - aufgrund weniger Strukturvorgaben                               & + aufgrund der erzwungenen Struktur                           \\
    schwerer große Projekte sauber zu & ergeben sich auch bei großen \\
    halten & Projekten wenig Probleme \\
    \hline
    \end{tabular}}
    \end{table}

React ist sehr weit verbreitet und wird von vielen Firmen genutzt.
Daher wollte unser Team zunächst diese etwas weiter verbreitete Library lernen. 
Außerdem ist React im Vergleich zu Angular noch etwas einfacher zu lernen.
Daher fiel auch die Wahl auf React, trotz der sehr NestJS-ähnlichen Struktur die Angular bietet.


\begin{table}[!h]
    \centering
    \resizebox{\textwidth}{!}{\begin{tabular}{|l|l|}
    \hline
    \textbf{MongoDB}                                                                      & \textbf{MySQL}                                                                    \\ 
    \hline
    + No SQL Datenbank & - SQL Datenbank               \\ 
    \hline
    + Datenstrukturen sind näher an                 & - Datenstrukturen sind abstrakt,  \\
    tatsächlicher Implementation & nicht so nah an der Implementation \\
    \hline
    + Speichert Daten im JSON Format   & - Speichert Daten in Tabellen        \\
     $\rightarrow$ Struktur nah an Objekten & \\
    \hline
    - Für uns neue Technologie                   & + Schon bekannt aus vorherigen Semestern                                  \\ 
    \hline
    \end{tabular}}
    \end{table}

Auch MongoDB war eine Vorgabe des ProductOwners.
Im Vergleich zu “klassischen” Datenbanken wie MySQL halten Mongo Datenbanken ihre Daten nicht in Tabellen, sondern in JSON Files in einer Baumstruktur, d.h. MongoDB ist eine sog. NoSQL Datenbank. \\
Daher ist MongoDB im Vergleich zu bspw. MySQL Datenbanken schon von der Grundstruktur der Datenhaltung sehr nah an der tatsächlich in der Entwicklung genutzten Strukturen der Implementation und verlangt so weniger Abstraktion als eine SQL Datenbank.
\pagebreak
\section{Softwarearchitektur}

\begin{figure}[!h]
    \centering
    \includegraphics[width=0.8\textwidth]{./UML_Diagrams/ComponentDiagramClient.png}
    \caption{Component Diagram: Client}
    \label{fig:ComponentDiagramClient}
\end{figure}
In Abbildung 4.1 ist das Komponentendiagramm des Client zu sehen.
Dieses beschreibt die verwendeten Komponenten bei eingegangenem User Input.
Der Nutzer macht über den Web-Client eine Anfrage, wie beispielsweise der Aufruf der Web-Applikation oder die Anfrage auf einen bestimmten Raum.
Dadurch fordert der Client ein Interface über die React App, welche dann eine REST Anfrage an die Nest API schickt.
Im Normalfall sollte dann die REST Anfrage beantwortet werden und die React App kann mit den erhaltenen Informationen ein geeignetes User Interface zur Nutzung auf dem Client bereitstellen.

\begin{figure}[!h]
    \centering
    \includegraphics[width=0.8\textwidth]{./UML_Diagrams/ComponentDiagramServer.png}
    \caption{Component Diagram: Server}
    \label{fig:ComponentDiagramServer}
\end{figure}
In Abbildung 4.2 ist das Komponentendiagramm des Servers zu sehen.
Dieses beschreibt die verwendeten Komponenten bei eingegangener REST Anfrage.
Die React App stellt eine REST Anfrage an die NEST API, diese fordert dann die verlangten Informationen über ein Interface von der Mongo Datenbank.
Die MongoDB ist erreichbar über den Port 27013 durch den Datenbank-Administrator, dieser bearbeitet die Anfrage und sendet die gefordeten Daten über das Interface zurück an die Nest API.
Zu guter Letzt kann die Nest API dann die bereitgestellten Informationen über ein geeignetes Interface an die React App schicken, welche dadurch weitere Anfragen, wie mithilfe des Komponentendiagramm des Clients beschrieben wurde, beantworten kann.



\begin{figure}[!h]
    \centering
    \includegraphics[width=0.8\textwidth]{./UML_Diagrams/Verteilungsdiagramm.png}
    \caption{Verteilungsdiagramm}
    \label{fig:Verteilungsdiagramm}
\end{figure}

Das in Abbildung 4.3 gezeigte Verteilungsdiagramm veranschaulicht, wie die Applikation im Browser des Nutzers über das React-Framework an das NestJS backend verbunden ist.
Der Nutzer sendet mithilfe der Benutzeroberflächen eine HTTP Anfrage an das React Front-End über das Internet, um zum Beispiel Informationen über Räume anzufordern.
Das React Front-End sendet die Anfrage des Nutzers mittels REST Anfrage über HTTPS an das NestJS Backend, damit dies an die Datenbank zugreifen kann.
Das NestJS Backend sendet die Anfrage an die Mongoose Datenbank und liefert die geforderten Informationen zurück.

\pagebreak
\section{Datenbanken}
\begin{figure}[!h]
    \centering
    \includegraphics[width=1\textwidth]{./images/EntityRelationshipModel.png}
    \caption{Entity-Relationship-Modell}
    \label{fig:EntityRelationshipModel}
\end{figure}
Der Mitarbeiter hat eine Rolle. 
Unter Rollen versteht sich die Position im Betrieb, darunter zählt der Geschäftsleitung, der Administrator oder der Mitarbeiter.
Die Personalnummer, der AnzeigeName und der LoginName sind zusätzliche Attribute.
Das Attribut aktiveChats steht für die Mitarbeiterchats, aktivBuchungen symbolisiert die getätigten Buchungen.
Die Buchungen haben als Attribut Datum, Startzeit und EndZeit.
Der Arbeitsplatz hat die Attribute Buchungsstatus, Koordinaten, die Räumlichkeiten abbilden, Buchungseinschränkung. 
Zudem gibt es Raum, Stockwerk und Gebäude, die den gebuchten Ort beschreiben. 
Ein Mitarbeiter kann mehrere Arbeitsplätze zu verschiedenen Daten, an verschiedenen Örtlichkeiten gleichzeitig buchen. 


% Bibliography
% \bibliography{file.bibtex}
\bibliographystyle{plain}

% Ending of document
\end{document}
